%
% File: chap01.tex
% Author: Victor F. Brena-Medina
% Description: Introduction chapter where the biology goes.
%
\let\textcircled=\pgftextcircled
\chapter{ONE}
\label{chap:intro}

\initial{A}griculture is the oldest profession. Humans started cultivation science even there was no civilization.
plant is a living thing that grows in earth, in water, or on other plants, usually has a stem, leaves, roots, and flowers, and produces seeds[Cambridge dictionary]
They provide the foundation of many food webs and animal life would not exist if plants were not around. 
With the development of science, plants were identified as living-thing. It could also respire, reproduce, and even get prone to various diseases. These are different types of diseases by various microorganisms may it be bacteria, viruses, or fungi. Plant diseases can damage crops to a great extent Causing losses in terms of production, both quantitative and qualitative, that clearly affect the farmer's revenues.
Generally, in some countries, a large team of experts will classify the disease with the naked eye. In some countries, Farmers are unaware of the disease and lack facilities to contact specialists. Under these conditions, they face many problems, including timing and access to high operating costs. Automatic detection of disease symptoms will improve adequate and cheaper than expert advice.

%=======
\section{Plant leaf diseases}
\label{sec:sec01}

The severity of plant disease ranges from minor damage of plant parts (leaves, stems, fruits) to total plant extinction. They affect plants by reducing their vigor, causing Reduced production quality and quantity, resulting in lower farmer income Reduced availability and higher prices of low-quality food for final consumers.
 Agricultural production, even leading to famine when attacking important large crops. Famine then leads to human suffering and diseases. People who are undernourished and weak from hunger are easy prey to cholera, pneumonia, stomach disorders, and other infectious diseases and parasites.\cite{book1} \\
 
 Three components are absolutely necessary in order for a disease to occur in any plant system. The three components are:
 \begin{enumerate}
     \item a susceptible host plant
     \item a virulent pathogen
     \item a favorable environment
 \end{enumerate}
 If any of the 3 factors is missing disease will not develop.
 \begin{figure}[!h]
     \centering
     \includegraphics[width=0.5\textwidth]{chapters/chapter01/fig01/triangle.jpg}
     \caption{Caption}
     \label{fig:my_label}
 \end{figure}

Plant diseases are classified as infectious and non-infectious depending on the nature of a causative agent.


\subsection{Infectious  plant diseases }
\label{subsec:subsec01}

Infectious  plant diseases can be figure as physiological or structural unrest caused by living (biotic) agents called "pathogens". These pathogens can be spread from an infected plant to a healthy plant.
The microorganisms that cause plant disease include  \textbf{nematodes, fungi, bacteria and viruses} . We also classify viruses as biologic because they must There are living cells that can reproduce and are composed of nucleic acids and proteins.\cite{book1}

\subsubsection{Nematodes}
\label{subsubsec:subsubsec01}

nematodes are small worm lives in the soil and feed on plant roots, although some species attacks aboveground plant parts and cause damage to plants by injuring cells, removing cell contents, or changing normal plant growth processes, Their annual global losses are estimated at 100 billion dollars.\cite{book1}

%A figures matrix.
\begin{figure}[!t]
\centering
\begin{minipage}{14cm}
    \centering
    \subtop[]{\includegraphics[width=0.6\textwidth]{fig01/Nématodes-1(nematodes knot the roots)}\label{sf:multiRH02d}}
\end{minipage}
\\ \vspace{0.1cm}
\begin{minipage}{15cm}
    \centering
    \subtop[]{\includegraphics[width=0.6\textwidth]{fig01/Nématodes-2}\label{sf:multiRH02d}}
\end{minipage}
\\ \vspace{0.1cm}
\caption{Nematodes}
\label{fig:Nematodes}
\end{figure}

\subsubsection{Fungal Diseases}
Fungi are small threadlike organisms composed of tiny filaments They cannot be seen without a microscope. Fungi may evolve into colonies similar to what we see on bread or vegetables, which improperly store for a long time ,examples :(downy and powdery mildew,
early and late blight ,Botrytis rots, Anthracnose, and Fusarium). Some fungi grow in plant tissues and destroy it by producing toxins and enzymes ,others develop into large Structures such as mushrooms.\cite{book1}
\begin{figure}[!h]
    \centering
    \includegraphics[width=0.9\textwidth]{chapters/chapter01/fig01/downy-mildew.jpg}
    \caption{Downey Mildew VS Powdery Mildew}
    
\end{figure}
\newpage
Most fungi produce microscopic spores  that spread frequently From one plant to another by wind, water or people. The main symptoms of fungal diseases include wilting, spotting, mold, overgrowth, deformations, mummification (shrinkage, darkening, and compaction of the infected tissue), and rot \cite{art2}
\begin{figure}[!h]
    \hspace{2.1cm}
    \includegraphics[width=0.25\textwidth]{chapters/chapter01/fig01/early-blight.jpg}
    \hspace{3cm}
    \includegraphics[width=0.38\textwidth]{chapters/chapter01/fig01/late-blight.jpg}
    \caption{Early blight VS Late blight }
    \label{fig:my_label}
\end{figure}

\subsubsection{Bacterial diseases}
Bacteria are small, single-celled organisms that possess rigid cell walls. Bacterial cells It must be zoomed approximately 400 times to be seen through a light microscope. \\
Bacteria reproduce by a process called binary fission in which one cell divides into two cells, Under suitable conditions  some bacteria can divide about every 30 min, The rapid multiplication of bacteria and the production of toxins and enzymes that destroy plant tissues contribute to the damage caused by bacteria.\cite{book1} \\
Examples of bacterial diseases :  Granville wilt , fire blight, blight of beans, soft rot crown gall, aster yellows and others.
\begin{figure}[!h]
    \centering
    \includegraphics[height=8cm]{chapters/chapter01/fig01/Granville_Wilt.jpg}
    \caption{Granville Wilt}
    \includegraphics[height=6cm]{chapters/chapter01/fig01/fire-blight.jpg}
    \caption{Fire blight}
\end{figure}
\subsubsection{Viral Diseases}
Viruses are composed of a nucleic acid molecule and a protective protein coat (capsid). The typical size of a plant virus is 30 nm They can be seen only under an electron microscope at Magnifications nearly to 2500 times \cite{art1} \\

The RNA in viruses directs certain enzyme systems in plant cells to manufacture components used to produce more virus particles.
In order to diverting energy and structural components,
normally used in plant growth, into reproductive processes for the virus. Viruses are
spread from one plant to another by insect or nematode vectors or by humans. \cite{book1} \\

The symptoms of viral diseases can be divided as follows: 
\begin{itemize}
    \item deformations (leaf wrinkling, threadlike leaves)
    \item growth suppression (reduced growth of the entire plant or its leading shoots)
    \item discoloration (mosaic, leaf chlorosis, variegation)
    \item necrosis
    \item impaired reproduction (flower sterility, parthenocarpy, shedding of flowers and ovaries)
\end{itemize}

In the majority of subtropical and tropical regions, a viral infection can lead to a loss of up to 98\% of the crop. \cite{art3} \\

There are many viral diseases among them Tobacco mosaic virus (TMV), Tomato spotted wilt virus (TSWV), Tomato yellow leaf curl virus (TYLCV), Cucumber mosaic virus (CMV), Potato virus Y (PVY) and others.

\begin{figure}[!h]
    \centering
    \includegraphics[width=0.55\textwidth]{chapters/chapter01/fig01/tswv.jpg}
    \caption{Tomato spotted wilt virus}
    \label{fig:my_label}
\end{figure}
\newpage
\begin{figure}[!h]
    \centering
    \includegraphics[width=0.7\textwidth]{chapters/chapter01/fig01/pvy.jpg}
    \caption{Potato virus Y}
    \label{fig:my_label}
\end{figure}

\subsection{Non Infectious diseases }
Non Infectious  plant diseases are caused by non living (abiotic) agents that cannot be transmitted from one plant to another. \cite{book1} These agents are usually: 
\begin{itemize}
    \item Environmental( moisture, soil compaction, ice, and sun scorch)
    \item nutritional(Calcium deficiency, nitrogen and iron deficiency )
    \item chemical factors (Chemical injuries due to misuse of insecticides, fungicides, and plant-growth regulators or their use in high or wrong rates).
\end{itemize}
\begin{figure}[!h]
    \centering
    \includegraphics[width=0.45\textwidth]{chapters/chapter01/fig01/calcium_deficiency.jpeg}
    \caption{Calcium deficiency}
\end{figure}
\newpage
Probably the major causes of poor health in plants are noninfectious diseases and his Symptoms appear suddenly, almost at once ,to their full intensity usually not progressive.
\begin{figure}[!h]
    \centering
    \includegraphics[width=0.7\textwidth]{chapters/chapter01/fig01/Scorch.jpg}
    \caption{Sun scorch}
    \label{fig:my_label}
\end{figure}
\section{Tomato production and statistics }
World tomato production is 120 Mt, of which one third in Asia, one third in Europe, one third in North America. There are now more than 500 varieties of tomatoes.\\

World tomato production is steadily increasing, rising from 64 million tonnes in 1988 to more than 100 million today, 30 million of which are intended for processing.\\

World production has increased by 35\% over the last ten years and is distributed as follows: Asia 45\%, Europe 22\%, Africa 12\%, North America 11\%, South America South and Central 8\%\\

According to statistics from the Food and Agriculture Organization of the United Nations, world tomato production in 2007 amounted to 126.2 million tonnes for an area of 4.63 million hectares, i.e. a average yield of 27.3 tonnes per hectare. 

%=========================================================